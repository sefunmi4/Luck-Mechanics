\documentclass[12pt]{article}
\usepackage{amsmath, amssymb, amsthm}
\usepackage{geometry}
\geometry{margin=1in}

\title{Research Proposal:\\
Luck Mechanics as Probabilistic Wavefunction Algebra and Geometry}
\author{Sefunmi Ashiru}
\date{\today}

\begin{document}
\maketitle

\begin{abstract}
This research program develops a unified framework called \emph{Luck Mechanics}, which treats chance as a quantifiable probabilistic field embedded in both mathematical and physical structures. The central thesis is that "luck" can be formalized as interference of probabilistic wavefunctions, compactified infinities, and harmonic arithmetic symmetries. The framework is explored through seven interlinked papers:
\begin{enumerate}
    \item Compactification of infinity (limits to complex infinities),
    \item Prime periodicity as sinusoidal interference (wavefunction sieve),
    \item Prime numbers as superposed wavefunctions (harmonic coverage \& computation),
    \item Attention and frequency models for AI and human consciousness,
    \item Polar ripple mapping of quantum mechanics,
    \item Spacetime as polar optics (gravity as interference),
    \item A flagship synthesis paper tying all components together.
\end{enumerate}
Each paper develops its own mathematical proofs, applications, and testable hypotheses, while contributing to the overall claim: \textbf{Luck is a quantifiable field spanning physics, mathematics, and cognition.}
\end{abstract}

\newpage
\tableofcontents
\newpage

% ====================================================
\section{Overarching Thesis}
Luck = (Opportunity $\times$ Preparation\_Action) + Prepared Circumstance.
\[
\mathcal{L}(x,t) = (\mathcal{O}(x,t)\cdot \mathcal{P}(x,t)) + \mathcal{C}(x,t).
\]
Here:
\begin{itemize}
  \item $\mathcal{O}$ = opportunity field (external probabilistic events),
  \item $\mathcal{P}$ = preparation-action field (internal readiness, amplification),
  \item $\mathcal{C}$ = prepared circumstance (bias or baseline field).
\end{itemize}
$\mathcal{L}(x,t)$ is sinusoidal with bias, representing constructive or destructive interference of chance. 
Luck cones = overlapping light cones of probability.

---

\section{Paper 1: Limits to Complex Infinities (Compactification Engine)}
\subsection{Core Claim}
Infinity can be compactified on curved surfaces, ensuring finite yet detailed mapping. This underwrites probability fields with stable boundaries.  

\subsection{Mathematics}
\[
f: \mathbb{R} \to S^1, \quad f(x)=(\cos(\alpha x),\sin(\alpha x)), \quad \alpha = \tfrac{1}{R}.
\]
\[
\Phi(z) = \Big( \tfrac{2\Re(z)}{|z|^2+1}, \tfrac{2\Im(z)}{|z|^2+1}, \tfrac{|z|^2-1}{|z|^2+1} \Big).
\]
- $\Phi$ = stereographic projection mapping $\mathbb{C}\cup\{\infty\}$ to Riemann sphere.  
- Compactification makes $+\infty$ and $-\infty$ continuous endpoints.  
- $\pi$ as scaling constant ensures irrational spacing → dense mapping.  

---

\section{Paper 2: Wavefunction Sieve of Eratosthenes (Prime Periodicity as Interference)}
\subsection{Core Claim}
Primes are surviving amplitudes after interference of periodic waves.  

\subsection{Mathematics}
For prime $p$:
\[
\psi_p(n) = \sin\!\Big(\frac{2\pi n}{p}\Big).
\]
Composite-sieve wavefunction:
\[
\Psi_P(n) = \prod_{\substack{p\ \text{prime}\\p \leq P}} \sin\!\Big(\frac{2\pi n}{p}\Big).
\]
Properties:
\begin{itemize}
\item If $n$ is composite, $\Psi_P(n)=0$.  
\item If $n$ is prime, $\Psi_P(n)\neq 0$.  
\end{itemize}

---

\section{Paper 3: Prime Numbers as Superposed Wavefunctions (Harmonic Coverage \& Computation)}
\subsection{Core Claim}
Primes behave as fundamental harmonics; coverage ratios describe their rarity.  

\subsection{Mathematics}
Define cosine prime wave:
\[
\W_p(n)=\cos\!\Big(\tfrac{2\pi n}{p}\Big).
\]
Total superposition:
\[
\W_{\text{tot}}(n)=\sum_{p\in \Primes} \W_p(n).
\]
Coverage ratio:
\[
f(p) = \frac{1}{p}\prod_{q<p}\Big(1-\frac{1}{q}\Big).
\]
Interpretation:
\begin{itemize}
\item Larger primes contribute smaller $f(p)$.  
\item Rareness $\sim 1/\ln p$ (Prime Number Theorem).  
\end{itemize}

---

\section{Paper 4: Attention \& Frequency Models (AI \& Consciousness)}
\subsection{Core Claim}
Brains and AI attention behave as probabilistic frequency filters, amplifying selected chance amplitudes.  

\subsection{Mathematics}
Attention operator:
\[
\mathcal{A}[f](t)=\int K(\omega)\,\hat f(\omega)e^{i\omega t}\,d\omega,
\]
where $K(\omega)$ is an attention kernel.  
Luck in cognition:
\[
\mathsf{Luck}_{\text{cog}}(t) = \big| \langle \Psi | \mathcal{A} | \Psi \rangle \big|^2.
\]
Interpretation:
- Attention = phase-selective amplification.  
- Consciousness = probabilistic resonance in frequency space.  

---

\section{Paper 5: Polar Ripple Framework (Radial–Angular Quantum Representation)}
\subsection{Core Claim}
Quantum states can be mapped as polar ripples with $r=ct$, $\theta$ = space.  

\subsection{Mathematics}
Coordinate transform:
\[
r=ct,\qquad \theta(x)=2\pi\frac{x+L/2}{L}.
\]
Polar wavefunction:
\[
\tilde{\Psi}(r,\theta) = \sqrt{\tfrac{2\pi}{L}}\,\Psi(x(\theta),t).
\]
Luck cones = overlapping angular ripple sectors.  

---

\section{Paper 6: Spacetime as Polar Optics (Unification Ansatz)}
\subsection{Core Claim}
Gravity emerges from overlapping quantum phases on a radial–angular manifold.  

\subsection{Mathematics}
Action:
\[
S[\A,g] = \int d^4x \sqrt{-g}\Big[ \tfrac{1}{2\kappa}R + \tfrac{\xi}{2}\nabla_\mu \A\nabla^\mu \A^* - V(|\A|)\Big].
\]
Stress-energy of phase:
\[
T_{\mu\nu}^{(\phi)}=\xi\big(\nabla_\mu \phi\nabla_\nu \phi - \tfrac{1}{2}g_{\mu\nu}(\nabla \phi)^2\big).
\]
Particles = compactified standing modes:
\[
n\lambda = L_\theta(r), \quad k_n=\tfrac{2\pi n}{L_\theta(r)}.
\]

---

\section{Paper 7: Flagship Synthesis}
\subsection{Goal}
Unify compactification, prime interference, harmonic computation, polar ripple quantum mechanics, and polar optics gravity into one coherent thesis.  

Main claim:  
\[
\textbf{Luck is a measurable field of probabilistic interference, spanning mathematics, physics, and cognition.}
\]

---

\section{Expected Outcomes}
\begin{enumerate}
    \item A compactification-based theory of infinite limits.  
    \item A sinusoidal interference model of prime rarity.  
    \item A harmonic superposition method for computation.  
    \item A frequency-attention model for AI and brains.  
    \item A polar coordinate representation for quantum mechanics.  
    \item A polar optics unification of gravity and quantum fields.  
    \item A flagship synthesis tying all results into ``Luck Mechanics''.  
\end{enumerate}

\end{document}