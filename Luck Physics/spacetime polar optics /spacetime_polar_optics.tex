% A 4D Radial–Angular Geometry for Unifying Gravity and Quantum (with assessment & citations + added refs)
\documentclass[reprint,amsmath,amssymb,aps,pra]{revtex4-2} % chktex 8

% ---------- Packages ----------
\usepackage{graphicx}
\usepackage{bm}
\usepackage{hyperref}
\usepackage{mathtools}
\usepackage{siunitx}
\usepackage{physics}
\usepackage{xcolor}
\usepackage{amsthm}
\usepackage[numbers]{natbib}

% Resolve siunitx/physics package command overlap
\AtBeginDocument{\RenewCommandCopy\qty\SI}

% ---------- Macros ----------
\newcommand{\Primes}{\mathbb{P}}              % Set of primes
\newcommand{\M}{\mathcal{M}}                   % Spacetime manifold
\newcommand{\g}{g}                             % Metric
\newcommand{\A}{\mathcal{A}}                   % Complex "light" field
\newcommand{\T}{T}                             % Stress-energy (generic)
\newcommand{\R}{\mathcal{R}}                   % Phase-tension scalar
\newcommand{\polang}{\vartheta}                % Angular coordinate
\newcommand{\polangb}{\boldsymbol{\vartheta}}  % Angular 2-sphere coords
\newcommand{\Eavg}{\mathcal{E}_{\!\mathrm{avg}}}
\newcommand{\taufield}{\mathsf{t}}             % tension density symbol
\newcommand{\Assessment}[1]{\par\smallskip\noindent\textit{\color{blue}[Assessment:\ #1]}\par\smallskip}

\hypersetup{colorlinks=true,linkcolor=blue,citecolor=blue,urlcolor=blue}

% ---------- Document ----------
\begin{document}

\title{Spacetime as Polar Optics: A 4D Radial-Angular Geometry for Unifying Gravity and Quantum}
\author{Oluwaseunfunmi Ashiru}
\affiliation{Theoretical Systems and Quantum Geometry, Poughkeepsie, NY}
\date{August 2025}

\begin{abstract}
  This paper aims to formalize a unification ansatz in which both gravitational dynamics and quantum phenomena inhabit the \emph{same} four-dimensional geometry: time is represented as a radial coordinate $r = ct$ emanating from a start event, while distinct ``versions'' (appearances) of that event correspond to different angular perspectives $\theta \in [0,2\pi)$ of a single history. A universal complex field $\mathcal{A} = |\mathcal{A}| e^{i\phi}$ models compactified standing-frequency excitations (``light'', broadly speaking) whose phase gradients generate stress and whose stress sources curvature through Einstein's equations. Measurement is recast as selecting frequency/phase within an angular kernel, so Born weights appear as geometric \emph{angular} weights on 4D light-cone slices. We articulate \emph{luck as polar optic mechanics}: apparent luck is angular/phase selection of a deterministic 4D propagation under finite signal speed. 
  We provide a common action, a polar metric adapted to causal flow, a prime-indexed standing-mode ansatz yielding compounded (composite) tension on each ``ring of time'', and a mapping between Feynman graphs~\cite{FeynmanHibbs1965} and 4D tension networks. For each major claim we add a rigorous status assessment vs.~modern GR, QFT, LQG, and String Theory, cite peer-reviewed literature, and include short proofs and test proposals.
\end{abstract}

\maketitle

% ============================================================
\section{Principle: One Geometry, Two Theories}
\textbf{Hypothesis.}
(i) Spacetime $(\M,\g_{\mu\nu})$ is the \emph{same} stage for gravity and quantum kinematics; (ii) use a radial-angular chart $(r,\polangb)$ adapted to causality with $r\equiv ct\ge 0$ and $\polangb\in S^2$ labeling ``perspectives'' (Bondi-Sachs-type null/angular foliations justify this viewpoint near light cones); (iii) distinct observed ``versions'' of events are different $\polangb$-slices of one 4D history; (iv) a universal complex field $\A$ underlies compactified standing modes.

\Assessment{Aligned: One 4D Lorentzian manifold for both GR dynamics and quantum kinematics is standard (quantum fields on curved backgrounds)~\cite{BirrellDavies,ParkerToms}. Using light-cone-adapted angular coordinates is orthodox (Bondi-Sachs framework)~\cite{Bondi1962,Sachs1962}. Novel: interpreting Born weights as \emph{angular} weights and casting ``luck'' as polar optics. Determinism in GR is subtle (global hyperbolicity, Cauchy horizons)~\cite{SmeenkWuthrich2021}. Constraint: any hidden-variable determinism must respect Bell tests~\cite{Hensen2015,Giustina2015,Shalm2015}; Bohm’s nonlocal theory is a classic example~\cite{Bohm1952}.}

% ============================================================
\section{Radial-Angular Metric and Causality}
Adopt a polar-like line element centered on a reference event:
\begin{equation}
  ds^2 = c^2\,dr^2 - a^2(r,\polangb)\,d\Omega^2, 
  \qquad d\Omega^2 = \gamma_{AB}(\polangb)\,d\polang^A d\polang^B,
  \label{eq:polar-metric}
\end{equation}
with areal factor $a(r,\polangb)$ encoding curvature/anisotropy; in flat space $a\to r$. This is consistent with light-cone/retarded-time (\`a la Bondi-Sachs) foliations used in gravitational radiation theory~\cite{Bondi1962,Sachs1962}. Radial nulls ($ds^2=0$) trace ordinary light cones; see also analyses of null cones in Minkowski backgrounds for GR reconstructions~\cite{PittsSchieve2004}.

\Assessment{Aligned (as a coordinate choice). Globally, Eq.~\eqref{eq:polar-metric} is not unique nor always regular, but null-foliation/angle charts are standard near null infinity and in wave zones~\cite{Will2014}.}

% ============================================================
\section{Unified Action and Phase Tension}
Let $\A=|\A|e^{i\phi}$ with action
\begin{equation}
  S[\A,\g] = \int d^4x\,\sqrt{-g}\Big[\frac{1}{2\kappa}R
    + \frac{\xi}{2}\,\nabla_\mu \A\,\nabla^\mu \A^* - V(|\A|)
    \Big],
  \label{eq:unified-action}
\end{equation}
yielding Einstein $G_{\mu\nu}=\kappa\, \T_{\mu\nu}^{(\A)}$ and a curved-space Klein-Gordon equation. Writing $\A=|\A|e^{i\phi}$ separates \emph{phase tension}
\begin{equation}
  \T_{\mu\nu}^{(\phi)} = \xi \Big(\nabla_\mu \phi\,\nabla_\nu \phi - \tfrac{1}{2} g_{\mu\nu} (\nabla \phi)^2 \Big),
  \label{eq:phase-tension}
\end{equation}
so phase gradients source curvature (akin to scalar-field stress in GR). String excitations interacting with strong gravitational waves exhibit resonant behavior~\cite{LiskaUnge2022}, supporting the broader wave/tension picture (though not this specific model).

\Assessment{Aligned: minimally coupled scalar fields sourcing curvature are textbook GR/QFTCS~\cite{BirrellDavies,ParkerToms}. Novel: treating \emph{all} matter as one complex ``light'' field departs from the Standard Model (SM); PDG reviews summarize the established multi-field SM~\cite{PDG2024}.}

% ============================================================
\section{Particles as Compactified Standing Modes}
On an angular loop at fixed $r$ with length $L_\polang(r)=\int\sqrt{a^2\,d\Omega^2}$, standing modes satisfy
\begin{equation}
  n\,\lambda = L_\polang(r),\qquad k_n=\frac{2\pi n}{L_\polang(r)},\ n\in\mathbb{N}.
  \label{eq:quantization}
\end{equation}
Mode energies $E_n=\hbar\omega_n$ follow local dispersion set by $V$ and curvature. As an analogy for spectral selection in structured media, quasiperiodic arrays exhibit localization bands and nontrivial selection rules~\cite{WahlstromChao1988}.

\Assessment{Aligned as a cavity/potential analogy. Novel if claimed literally for the SM spectrum; empirically, SM particle content \emph{is not} explained as compactified photon modes~\cite{PDG2024}.}

% ============================================================
\section{Born Weights as Angular Weights; Luck as Polar Optic Mechanics}
Define an angularly normalized field on each ring of time
\begin{equation}
  \tilde{\Psi}(r,\polangb) \equiv \sqrt{\frac{1}{\Omega_r}}\,\A(r,\polangb),\qquad
  \Omega_r\equiv\int a^2(r,\polangb)\,d\Omega.
  \label{eq:polar-psi}
\end{equation}
Detection with instrument kernel $K_\omega$ yields
\begin{equation}
  P(\omega,\polangb\mid r)\propto\left|\int d\Omega'\,K_\omega(\polangb,\polangb')\,\tilde{\Psi}(r,\polangb')\right|^2.
\end{equation}
\emph{Luck as polar optics:} apparent randomness reflects finite-$c$ access to phases from only part of $S^2$; probability is an angular under-sampling of an underlying interference field.

\Assessment{Complementary interpretive layer; compatible with decoherence/Born usage in quantum optics~\cite{MandelWolf1995}. Constraint: local hidden-variable completions are excluded by loophole-free Bell tests; any deterministic completion must be explicitly nonlocal/contextual~\cite{Hensen2015,Giustina2015,Shalm2015,Bohm1952}.}

% ============================================================
\section{Prime-Indexed Standing Modes \& Compounded Tension on the Ring of Time}
Let
  \begin{equation}
    f(\theta)=\sum_{n\ge1} a_n \cos{(n\theta+\varphi_n)},\quad
    \taufield(\theta)=\frac{\eta}{2}\big(\partial_\theta f\big)^2.
  \label{eq:fourier-tension}
\end{equation}
If we \emph{seed} only prime harmonics $p\in\Primes$, quadratic mixing and weak nonlinearities generate composite indices via sum/difference identities; see Appendix~\ref{app:primeproof}. A von~Mangoldt-weighted ``sieve'' operator,
\begin{equation}
\mathcal{S}[f](\theta)=\sum_{n\ge1}\Lambda(n)\, a_n e^{i(n\theta+\varphi_n)},
  \label{eq:vonmangoldt}
\end{equation}
diagnoses missing primes: adding a missing $p$ reduces composite-phase error at multiples of $p$. Average tension obeys
\begin{equation}
\Eavg[\taufield]=\frac{\eta}{4}\sum_{n\ge1} n^2 a_n^2,
  \label{eq:avg-tension}
\end{equation}
so rarity ($\sim 1/\ln p$) versus energy ($\propto p^2 a_p^2$) trade off (Prime Number Theorem background; spectral analogies with prime statistics are suggestive~\cite{Montgomery1973,BerryKeating1999}).

\Assessment{Aligned: Fourier analysis and nonlinear wave-mixing (sum/difference frequency generation) are standard in nonlinear optics/fluids~\cite{BoydNLO}. Novel: using prime seeding as a \emph{design ansatz} for compounded tension and ``prime events''.}

% ============================================================
\section{Tension Networks and Feynman Graphs}
Define the phase-tension scalar $\R = (\nabla \phi)^2$. High-$\R$ filaments define a graph $\mathcal{G}\subset\M$: edges follow large phase gradients, vertices are compactification knots, sheets are interference membranes. Feynman graphs then serve as bookkeeping shadows of stress/propagation on $\mathcal{G}$.

\Assessment{Heuristic mapping. Aligned in spirit with path-integral/diagrammatics~\cite{Feynman1948,Dyson1949}. Novel as a literal identification of spacetime tension filaments with diagrammatic edges.}

% ============================================================
\section{Black/White Holes as Condensates of the Same Jelly}
When phase tension and energy focus, $a(r,\polangb)$ shrinks and null congruences converge (black-hole trapping: Kerr etc.)~\cite{Kerr1963,Kraniotis2005}. The Kruskal extension of Schwarzschild includes a white-hole region as the time-reverse of a black hole~\cite{Kruskal1960}. Higher-dimensional analogs (5D black holes/rings) display rich surface geometry~\cite{FrolovGoswami2007}.

\Assessment{Aligned: black holes; white holes exist as time-reversed regions in extended solutions. Novel/speculative: ``mutual reopening/repulsion'' of nearby white-hole–like sources in nature. No observational support; GR allows white-hole regions mathematically but they are unstable/unobserved.}

% ============================================================
\section{Recovering Limits and Confrontation with Data}
\textbf{GR limit.} With weak $\phi$ gradients, $\T^{(\phi)}_{\mu\nu}$ is a standard scalar source and Einstein’s equations reduce to GR with known tests (perihelion advance, Shapiro delay, frame dragging, gravitational waves)~\cite{Will2014,GPB2011,LIGO2016,Kraniotis2005}.

\textbf{Quantum limit.} On fixed $(\M,\g)$, small-amplitude $\A$ obeys the curved-space KG equation~\cite{BirrellDavies,ParkerToms}; standard interference/Born rule are recovered.

\textbf{SM precision.} Any unification must not spoil QED’s precision (electron $g-2$ and tenth-order calculations)~\cite{Hanneke2008,Aoyama2012,Parker2018}. Our construction stays agnostic about SM gauge content (to be added atop $\A$ if needed). Historical alternatives like a ``neutrino theory of photons'' are noted but not compatible with the modern SM~\cite{Perkins1965,PDG2024}.

\Assessment{Aligned: these limits are standard. Constraint: extremely tight QED/GR tests bound any new couplings.}

% ============================================================
\section{Phenomenology and Tests}
\textbf{Analog fluids/photonics.} Drive superfluids or optical cavities with prime-indexed phase masks; verify composite lines appear by mixing and that adding a ``missing prime'' reduces sieve error (nonlinear mixing calibrated by~\cite{BoydNLO}).

\textbf{Interferometry as angular tomography.} Multi-aperture interferometers resolving $\polangb$ should exhibit curvature-dependent reweighting of fringe envelopes consistent with $a(r,\polangb)$ (a geometric calibration of angular Born weights).

\textbf{Hydrodynamic quantum analogs.} Examine whether droplet pilot-wave analogs show prime-seeded stability bands~\cite{CouderFort2006,Bush2015}.

\Assessment{Falsifiable lab signatures exist for the prime-sieve/tension story and for angular tomography. Astrophysical deviations from GR lensing would tightly bound any new phase-tension couplings.}

% ============================================================
\section{Status Map: What Aligns, What’s New, What’s Skeptical}
\begin{itemize}
\item \textbf{Fully aligned with modern science:} One 4D Lorentzian manifold; null/angle foliations (Bondi–Sachs); GR tests including frame-dragging and gravitational waves; QFT on curved spacetime; QED precision; Kerr/Schwarzschild/Kruskal basics~\cite{Will2014,LIGO2016,GPB2011,BirrellDavies,ParkerToms,Kerr1963,Kraniotis2005,Kruskal1960,PDG2024}.
\item \textbf{New but complementary (conceptual):} Born weights as angular weights; “luck as polar optics”; tension-network interpretation of diagrams; prime-seed design ansatz for compounded tension (intended for analog systems). Must be checked against Bell tests and precision constraints~\cite{Hensen2015,Giustina2015,Shalm2015,Bohm1952}.
\item \textbf{Far-fetched/ruled-out as stated:} ``Everything is compactified light’’ (conflicts with SM content~\cite{PDG2024}); white-hole repulsion in nature (no evidence; white holes are non-astrophysical regions of extended solutions); any \emph{local} hidden-variable determinism (excluded by loophole-free Bell tests~\cite{Hensen2015,Giustina2015,Shalm2015}). Historical proposals like a neutrino theory of photons are not consistent with current electroweak theory~\cite{Perkins1965,PDG2024}.
\end{itemize}

% ============================================================
\section{Conclusion}
Assuming one 4D geometry with time as radius and angles as perspectives produces a clean synthesis: the “material” of many apparent worlds is an oscillatory field observed through angular kernels; particles are standing-mode knots (in analogy), and Born weights can be recast as angular weights. We provided a scalar-GR action, a polar metric, a prime-sieve standing-mode mechanism that deterministically compounds into composite tension on each ring of time, and a tests list. The decisive next step is empirical: cavity/superfluid/photonic analogs and angular tomography that can confirm or falsify the prime-sieve and polar-optic predictions.

% ============================================================
\appendix

\section{Polar Normalization and Perspective Weights}\label{app:polar}
With $r=ct$ and $\polangb\in S^2$ with area element $a^2(r,\polangb)\,d\Omega$, define $\tilde{\Psi}$ by Eq.~\eqref{eq:polar-psi}. Then
\begin{align}
 \int_{S^2} |\tilde{\Psi}(r,\polangb)|^2\, a^2(r,\polangb)\, d\Omega
 &= \frac{1}{\Omega_r}\int_{S^2} |\A(r,\polangb)|^2 a^2\, d\Omega \;=\; 1,
\end{align}
so angular detection weights are properly normalized on each radial slice.

\section{Prime Mixing \& Average Tension}\label{app:primeproof}
  Let $f(\theta)=\sum_{n\ge1} a_n \cos{(n\theta+\varphi_n)}$. Then
  $\partial_\theta f=\sum_n (-n a_n)\sin{(n\theta+\varphi_n)}$ and
\[
    \taufield(\theta)=\tfrac{\eta}{2}(\partial_\theta f)^2
    = \tfrac{\eta}{2}\sum_{m,n} mn\,a_m a_n \sin{(m\theta+\phi_m)}\sin{(n\theta+\phi_n)}.
\]
  Using $\sin u\,\sin v=\tfrac{1}{2}[\cos{(u{-}v)}-\cos{(u{+}v)}]$, quadratic mixing generates indices $|m\pm n|$. If $f$ is seeded only at primes, composites appear in $\taufield$ automatically. Averaging over $\theta$ kills cross terms, yielding Eq.~\eqref{eq:avg-tension}.

\section{Bell Constraints on ``Polar Luck''}
Any claim that probabilities arise from inaccessible but \emph{local} angular phases is equivalent to a local hidden-variable model and violates loophole-free Bell inequalities~\cite{Hensen2015,Giustina2015,Shalm2015}. Any deterministic completion must be nonlocal/contextual (or accept retrocausality), which we leave outside scope.

% ============================================================
% ============================================================

\bibliographystyle{plainnat}
\bibliography{references}

\end{document}
