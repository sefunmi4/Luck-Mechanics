\documentclass[12pt]{article}

% ----------------------------------------------------
% PACKAGES
% ----------------------------------------------------
\usepackage{amsmath, amssymb, amsthm}
\usepackage{geometry}
\usepackage{graphicx}
\usepackage{hyperref}
\usepackage{natbib}
\geometry{margin=1in}

% ----------------------------------------------------
% META INFO
% ----------------------------------------------------
% Update these per paper
\title{[Working Title for Paper X]\\
Subtitle if needed}
\author{Sefunmi Ashiru}
\date{\today}

% ----------------------------------------------------
\begin{document}
\maketitle

% ----------------------------------------------------
\begin{abstract}
% Keep the abstract short (150–250 words).
% 1–2 sentences: motivation (why this problem matters).
% 1–2 sentences: method or approach (the math you introduce).
% 1–2 sentences: results and implications.
\end{abstract}

\newpage
\tableofcontents
\newpage

% ====================================================
\section{Introduction}
% - Introduce the field (math, physics, AI).
% - State the specific gap/problem.
% - Place your contribution in 1–2 clear sentences.
% - End with roadmap: “This paper is structured as follows…”

% Example phrasing template:
% The distribution of prime numbers has long fascinated mathematicians...
% In this paper, we reinterpret the classical sieve as a wavefunction interference problem.
% Our contributions are: (1) definition..., (2) proof..., (3) discussion...
% The paper is organized as follows: Section 2 reviews..., Section 3 introduces..., etc.
% ---------------------------------------------------

% ====================================================
\section{Related Work}
% - Tailor this section per discipline.
%   Math → cite sieve theory, PNT, zeta.
%   Physics → cite GR, QFT, wave interference.
%   AI → cite attention mechanisms, frequency analysis.
% - 3–6 key references are enough for a preprint.
% - Keep the tone respectful but concise: “Prior approaches have focused on..., here we propose...”
% ---------------------------------------------------

% ====================================================
\section{Mathematical Framework / Model}
% - This is the “proof core”.
% - Define symbols and functions carefully.
% - Copy the math skeleton from your proposal here.
% - Use Lemma / Theorem / Proof environments if useful.
%
% Example:
% \begin{theorem}
% For prime $p$, the function $\psi_p(n) = \sin(2\pi n/p)$ vanishes at all multiples of $p$.
% \end{theorem}
% \begin{proof}
% ...
% \end{proof}
% ---------------------------------------------------

% ====================================================
\section{Results and Applications}
% - Apply the framework to concrete cases.
% - For math papers: plots of prime density, coverage ratios.
% - For physics papers: diagrams of luck cones, ripple visualizations.
% - For AI/consciousness: toy model, frequency kernel.
% - Figures: always include 1–2 figures per paper, consistent style.
%
% Example:
% \begin{figure}[h]
%   \centering
%   \includegraphics[width=0.8\linewidth]{example_plot.png}
%   \caption{Example plot of prime wavefunction interference.}
%   \label{fig:primewave}
% \end{figure}
% ---------------------------------------------------

% ====================================================
\section{Discussion}
% - Interpret the results.
% - What is new here? Why does it matter to this field?
% - What are the limitations? (reviewers look for this).
% - Connect gently back to the “Luck Mechanics” thesis
%   but keep this paper self-contained.
% ---------------------------------------------------

% ====================================================
\section{Conclusion and Future Work}
% - Summarize contributions in 2–3 sentences.
% - Suggest next steps (e.g., extend to complex plane, test experimentally, integrate into physics).
% - End with one sentence that hints at connection to other Luck Mechanics papers.
% ---------------------------------------------------

% ====================================================
\bibliographystyle{plainnat}
\bibliography{references}
% Keep one shared .bib file across all 6 papers for consistency.
% ---------------------------------------------------

\end{document}