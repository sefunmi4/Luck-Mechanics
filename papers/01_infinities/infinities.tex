\documentclass[12pt]{article}

% ----------------------------------------------------
% PACKAGES
% ----------------------------------------------------
\usepackage{amsmath, amssymb, amsthm}
\usepackage{tikz}
\usepackage{geometry}
\geometry{margin=1in}
\usepackage{hyperref}

% ----------------------------------------------------
% THEOREM ENVIRONMENTS (use as needed)
% ----------------------------------------------------
\newtheorem{definition}{Definition}
\newtheorem{proposition}{Proposition}
\newtheorem{theorem}{Theorem}
\newtheorem{lemma}{Lemma}
\newtheorem{corollary}{Corollary}
\theoremstyle{remark}
\newtheorem*{remark}{Remark}
\newtheorem*{example}{Example}

% ====================================================
% VERSION NOTES (remove in camera-ready)
% ====================================================
% v2 EDITOR NOTES:
% 1) Add precise statements + short proofs for:
%    (i) One-point compactification: R ~ S^1, R^2 ~ S^2 via stereographic projection.
%    (ii) "Limit at infinity" <-> "continuous extension to compactification".
%    (iii) Irrational rotations on S^1: density/equidistribution (Weyl/Kronecker).
%    (iv) Complex case: classify behavior at infinity for meromorphic f (removable/pole/essential).
% 2) Keep "role of pi" claim mathematically careful: it's not "uniqueness", it's
%    "avoidance of short rational periods" and (under hypotheses) uniform angular distribution.
% 3) Include 2–3 figures: (A) Stereographic map for R ↔ S^1 and C ↔ \hat{\mathbb{C}},
%    (B) Spiral compactification on unit disk (to discuss ±∞ identification),
%    (C) Directional approach to ∞ on S^2 (arrows to North pole).
% 4) Add a brief Related Work paragraph citing standard texts (topology, complex analysis).
% 5) Ensure every informal statement has either a reference or a short proof sketch.



\begin{document}

% ----------------------------------------------------
% META
% ----------------------------------------------------
\title{Limits to Complex Infinities:\\ Mapping Infinite Numbers to Circles and Spheres}
\author{Daniel Valvo, Sefunmi Ashiru}
\date{\today}

\maketitle

\begin{abstract}
    The behavior of functions at infinity remains ambiguous in analysis, particularly over $\mathbb{R}^n$ where $+\infty$ and $-\infty$ may be treated as either distinct endpoints or compactified into a single point. This paper develops a geometric framework for resolving such ambiguities by mapping infinite values onto circles, spheres, and ultimately the Riemann sphere. The method employs $\pi$ as a natural angular scaling constant, avoiding rational periodic degeneracies and ensuring dense coverage under compactification. We first formalize mappings from $\mathbb{R} \to S^1$ and $\mathbb{R}^2 \to S^2$, then extend to $\mathbb{C}$ via stereographic projection, where infinity corresponds to the north pole. Within this framework, we establish conditions under which limits at $\pm\infty$ coincide, classify cases of continuous versus discontinuous infinity, and provide criteria for continuity of functions under compactification. The results yield a unified treatment of limits and convergence in both real and complex settings, while highlighting implications for number theory, modular arithmetic, and physical models with periodic boundary conditions.
\end{abstract}

\tableofcontents

% ====================================================
\section{Introduction}
Infinity is traditionally approached linearly along the extended real line 
$\overline{\mathbb{R}} = \mathbb{R} \cup \{\pm \infty\}$. In this model, $+\infty$ and 
$-\infty$ are treated as distinct endpoints, and limits are analyzed separately along 
each direction. However, this representation obscures deeper geometric structure. 
Compactification techniques provide an alternative: by embedding $\mathbb{R}$ or 
$\mathbb{R}^n$ into a closed manifold such as a circle or sphere, one may replace 
the “ends” of the line or plane with a single point at infinity. This reframing 
not only unifies the analysis of limits but also aligns naturally with the 
Riemann sphere construction in complex analysis, where $\infty$ appears as an 
ordinary point of the extended plane.  

The purpose of this paper is to develop a systematic framework for mapping 
infinite values to curved spaces and to analyze the implications for limits, 
continuity, and convergence. Our main contributions are as follows:  

\begin{enumerate}
    \item We formalize the compactification of $\mathbb{R}$ onto $S^1$ and 
    $\mathbb{R}^2$ onto $S^2$, establishing their equivalence to stereographic 
    projection models (Theorem~\ref{thm:compactification-equivalence}).  
    \item We prove that a function $f : \mathbb{R} \to \mathbb{R}$ admits a limit 
    at infinity if and only if it extends continuously to the added point under 
    circle or sphere compactification (Theorem~\ref{thm:limit-extension}).  
    \item We show that irrational angular scalings, such as those involving $\pi$, 
    produce dense or equidistributed images on $S^1$, thereby avoiding 
    short-period degeneracies that arise with rational multiples 
    (Proposition~\ref{prop:irrational-scaling}).  
    \item For the complex case, we use stereographic projection to identify 
    $\mathbb{C} \cup \{\infty\}$ with the Riemann sphere. We classify the behavior 
    of limits at infinity for meromorphic functions in terms of the image of the 
    north pole (Theorem~\ref{thm:riemann-meromorphic}).  
\end{enumerate}

Taken together, these results provide a unified treatment of infinity across real 
and complex settings, clarifying when infinity behaves as a single continuous 
point and when it must be considered in its directional form.  

\textbf{Paper roadmap.} Section~2 reviews related work in compactification and 
conformal geometry. Section~3 develops the mathematical framework and states the 
main theorems. Section~4 applies the framework to real, spherical, and complex 
cases with illustrative figures. Section~5 discusses implications and limitations, 
and Section~6 concludes with future directions.  

% ====================================================
\section{Preliminaries and Notation}
\label{sec:preliminaries}

% PURPOSE: Collect standard definitions so later proofs read smoothly.
% Balance rigor with intuitive explanations.

We begin by recalling standard notions from topology and analysis that will 
underpin the compactification framework. Whenever possible, we accompany the 
formal definition with an intuitive picture.

\begin{definition}[One-point compactification]
Let $X$ be a locally compact, non-compact Hausdorff space. The one-point 
compactification $X^* := X \cup \{\infty\}$ is obtained by adjoining a single 
“point at infinity” to $X$. A neighborhood of $\infty$ is defined to be the 
complement of a compact subset of $X$, while the open sets of $X$ remain 
unchanged.
\end{definition}

\begin{remark}[Intuition]
On the real line $\mathbb{R}$, this means we bend the line into a circle by 
gluing the two “ends” ($+\infty$ and $-\infty$) together into a single new point. 
This is why the one-point compactification of $\mathbb{R}$ is homeomorphic to a 
circle $S^1$, and more generally $\mathbb{R}^n$ compactifies to the $n$-sphere $S^n$. 
This matches the geometric idea of “closing off” infinity by wrapping the space 
onto a sphere.
\end{remark}

\begin{definition}[Limit at infinity on $\mathbb{R}^n$]
For $f:\mathbb{R}^n \to \mathbb{R}^m$, we write 
\[
\lim_{\|x\|\to \infty} f(x) = L
\]
if for every $\varepsilon > 0$ there exists $R > 0$ such that 
$\|x\| > R \implies \|f(x)-L\| < \varepsilon$. 
\end{definition}

\begin{remark}[Intuition]
This definition says: as you go further and further away from the origin in 
$\mathbb{R}^n$, the outputs of $f$ settle arbitrarily close to $L$. The compactified 
picture replaces “going further and further” with “approaching the added point at infinity.”
\end{remark}

\begin{definition}[Stereographic projection]
The stereographic projection $\sigma : S^n \setminus \{N\} \to \mathbb{R}^n$ 
is the map obtained by projecting from the north pole $N$ of the $n$-sphere $S^n$ 
onto the equatorial hyperplane. Its inverse extends $\mathbb{R}^n$ by sending the 
“point at infinity” to $N$. For $n=2$, this identifies the extended complex plane 
$\hat{\mathbb{C}} = \mathbb{C}\cup\{\infty\}$ with the Riemann sphere.
\end{definition}

\begin{remark}[Accessible picture]
Imagine placing a unit circle so that it sits on the number line, centered at 0. 
Pick a point $x$ on the line, and draw a straight line from the top of the circle 
(the point at angle $\pi/2$) through $x$. That line hits the circumference at 
a unique point, which we take as the projection of $x$. This triangle construction 
gives a continuous way to map the entire line onto the circle, with the 
point at infinity corresponding to the top of the circle. In higher dimensions, 
the same idea projects $\mathbb{R}^n$ onto a sphere $S^n$.
\end{remark}

\begin{definition}[Conformal property]
For $n=2$, stereographic projection is conformal: it preserves angles between 
curves. In particular, straight lines and circles in $\mathbb{C}$ correspond to 
circles on $S^2$ (possibly through the point at infinity).
\end{definition}

\begin{remark}
This property explains why the Riemann sphere is so natural in complex analysis: 
it allows us to treat infinity as just another point, without breaking the 
geometric rules for angles.
\end{remark}

\begin{definition}[Irrational rotation on the circle]
Let $\alpha \in \mathbb{R} \setminus \mathbb{Q}$. The map 
$R_\alpha : S^1 \to S^1$, defined by 
\[
R_\alpha(\theta) = \theta + 2\pi \alpha \pmod{2\pi},
\]
generates a dense orbit on $S^1$. In contrast, rational rotations yield finite 
cyclic orbits.
\end{definition}

\begin{remark}[Connection to $\pi$]
Choosing $\pi$ (or any irrational multiple) as a scaling constant ensures that 
points mapped from $\mathbb{R}$ wrap around $S^1$ without repeating in a short 
cycle. Instead, they fill the circle densely in a uniform fashion. This avoids 
degenerate periodicities and makes $\pi$ central to the compactification scheme.
\end{remark}

% References:
% - Munkres, *Topology*
% - Rudin, *Real and Complex Analysis*
% - Ahlfors, *Complex Analysis*
% - Walters, *An Introduction to Ergodic Theory* (irrational rotations).

% ====================================================
\section{Real Line Mapping to a Circle}
\label{sec:circle}

We first consider how to compactify $\mathbb{R}$ onto a circle $S^1$. There are 
several complementary constructions, each emphasizing different properties.  

% ---------------------------------------------------
\subsection{Periodic wrapping map}
\label{sec:wrap}
A natural first attempt is to wrap the line around the unit circle periodically:
\begin{equation}\label{eq:wrap}
f: \mathbb{R} \to S^1, 
\quad f(x) = (\cos(\alpha x), \sin(\alpha x)), 
\quad \alpha = \frac{1}{R}.
\end{equation}
For $R=1$, the mapping has period $2\pi$. 

\begin{proposition}[Irrational rotation avoids short-period degeneracy]
If $\alpha/(2\pi)\notin \mathbb{Q}$, then the orbit $\{e^{i\alpha n} : n \in \mathbb{Z}\}$ 
is dense in $S^1$. 
\end{proposition}

\begin{proof}[Proof sketch]
This is the classical Kronecker–Weyl equidistribution theorem: rational rotations 
yield finite orbits, while irrational rotations equidistribute on the circle.  
\end{proof}

\begin{remark}
This construction is not injective, since points differing by multiples of $2\pi/\alpha$ 
map to the same point. It is most useful for illustrating how irrational scalings 
such as $\pi$ prevent short cycles and create dense coverage.
\end{remark}

% ---------------------------------------------------
\subsection{Spiral embedding}
Another option is to embed $\mathbb{R}$ into the unit disk with a spiral:
\begin{definition}[Spiral compactification]
Define $h:\mathbb{R}\to \overline{\mathbb{D}}$ by
\[
h(x) = r(x)\,(\cos x, \sin x), \quad r(x) = 1-e^{-|x|}.
\]
Then $h$ is injective, and its image spirals toward the unit circle as 
$|x|\to\infty$. Identifying the entire boundary circle to a single point realizes 
the one-point compactification.
\end{definition}

This picture preserves injectivity on $\mathbb{R}$ but requires collapsing the 
boundary circle to unify $+\infty$ and $-\infty$.  

% ---------------------------------------------------
\subsection{Tangent half-angle compactification}
The canonical construction comes from the tangent half-angle formulas. Define
\begin{equation}\label{eq:tangent}
\phi:\mathbb{R}\to S^1, \quad
\phi(t) = \left(\frac{1-t^2}{1+t^2},\ \frac{2t}{1+t^2}\right), \quad t\in\mathbb{R}.
\end{equation}
This is equivalent to setting $\theta = 2\arctan(t)$ and using 
$(\cos\theta,\sin\theta)$. 

\begin{remark}[Geometric interpretation]
Imagine a unit circle centered at the origin. From the north pole $(0,1)$, draw a 
straight line through a point $(t,0)$ on the real axis. The intersection with the 
circle is $\phi(t)$. As $t\to\pm\infty$, the point approaches the north pole. Thus 
$\mathbb{R}\cup\{\infty\}\cong S^1$.
\end{remark}

\begin{theorem}[One-point compactification via tangent half-angle]\label{thm:tangent}
The map $\phi$ is a homeomorphism between $\mathbb{R}\cup\{\infty\}$ and $S^1$.  
\end{theorem}

\begin{proof}[Proof sketch]
Continuity and bijectivity follow from the tangent half-angle identities for 
$\sin\theta,\cos\theta$, with $\infty$ corresponding to $\theta=\pi$.  
\end{proof}

\begin{remark}[Contrast with other maps]
Unlike the periodic wrapping \eqref{eq:wrap}, $\phi$ is injective. Unlike the spiral, 
it is the canonical stereographic projection and thus standard in topology.  
\end{remark}

% ---------------------------------------------------
\subsection{Bias and focal rotation parameter}
To “focus attention” near a chosen point $b\in\mathbb{R}$, we introduce a bias 
parameter. Define
\begin{equation}\label{eq:bias}
\phi_b(t) = \left(\frac{1-(t-b)^2}{1+(t-b)^2},\ \frac{2(t-b)}{1+(t-b)^2}\right).
\end{equation}
Geometrically, this shifts the center of compactification so that $t=b$ is mapped 
to the south pole $(0,-1)$, and the “point at infinity” lies directly opposite.  

\begin{remark}[Interpretation]
- For $b=0$, we recover the symmetric compactification with detail near the origin.  
- For large positive $b$, the projection emphasizes structure near $x=b$ while 
sending both ends of the number line far away.  
- This gives a tunable way to zoom: we can view $\infty$ as the focus (north pole), 
or we can center the detail elsewhere on the line.  
\end{remark}

% ---------------------------------------------------
\subsection{Limits at infinity via circle compactification}
\begin{theorem}[Limit at infinity via circle compactification]\label{thm:RtoS1}
Let $f:\mathbb{R}\to\mathbb{R}^m$ be continuous. The following are equivalent:
\begin{enumerate}
    \item $\lim_{x\to\pm\infty} f(x)$ exists and the two one-sided limits agree.  
    \item $f$ extends continuously to a map $\tilde f:S^1\to\mathbb{R}^m$ under 
    the tangent half-angle compactification.  
\end{enumerate}
\end{theorem}

\begin{remark}[Distinct endpoints model]
If we instead use the extended line $\overline{\mathbb{R}}$ with two endpoints, 
the condition weakens: $\lim_{x\to +\infty}$ and $\lim_{x\to -\infty}$ may differ, 
corresponding to continuity at two separate points.  
\end{remark}

% ====================================================
\section{Extension to the Sphere}
\label{sec:sphere}

The one-dimensional circle compactification has a natural higher-dimensional 
analogue: wrapping the plane $\mathbb{R}^2$ onto the two-sphere $S^2$. 
This is realized by stereographic projection, which can be derived using 
the same tangent half-angle trick introduced in Section~\ref{sec:circle}.

% ----------------------------------------------------
\subsection{Stereographic projection from the plane}

\begin{definition}[Stereographic projection from $\mathbb{R}^2$ to $S^2$]
Define 
\begin{equation}\label{eq:stereographic}
\mathrm{St}:\mathbb{R}^2 \to S^2\setminus\{N\}, \qquad
(x,y)\mapsto \left(\frac{2x}{x^2+y^2+1},\ \frac{2y}{x^2+y^2+1},\ \frac{x^2+y^2-1}{x^2+y^2+1}\right),
\end{equation}
where $N=(0,0,1)$ is the north pole of the unit sphere $S^2$. Its inverse 
$\mathrm{St}^{-1}:S^2\setminus\{N\}\to\mathbb{R}^2$ maps a point on the sphere 
back to the plane.
\end{definition}

\begin{remark}[Geometric picture]
Place a sphere so that it rests on the plane at the south pole. For each 
$(x,y)\in\mathbb{R}^2$, draw a straight line from the north pole $N$ through 
$(x,y,0)$. This line intersects the sphere at a unique point, which is 
$\mathrm{St}(x,y)$. The “missing point” $N$ corresponds to the point at infinity.  
This directly generalizes the circle construction of Section~\ref{sec:circle}.
\end{remark}

% ----------------------------------------------------
\subsection{Tangent half-angle derivation}

In polar coordinates $(x,y) = (r\cos\varphi,r\sin\varphi)$, define the colatitude 
$\Theta$ (angle from the north pole) by the relation
\[
\tan\!\left(\frac{\Theta}{2}\right) = \frac{1}{r}.
\]
Using half-angle identities, one computes
\[
\sin\Theta = \frac{2r}{1+r^2}, \qquad \cos\Theta = \frac{r^2-1}{r^2+1}.
\]
Thus the projection point in Cartesian coordinates is
\begin{equation}
(X,Y,Z) = \left(\frac{2x}{1+r^2},\ \frac{2y}{1+r^2},\ \frac{r^2-1}{r^2+1}\right),
\end{equation}
which agrees with formula~\eqref{eq:stereographic}.  

\begin{remark}[Analogy with the circle case]
For large $r$, $\Theta \to 0$ and the point approaches the north pole.  
For small $r$, $\Theta \to \pi$ and the point lies near the south pole.  
This is exactly the 2D analogue of the tangent half-angle parametrization 
that compactified $\mathbb{R}$ to $S^1$.  
\end{remark}

% ----------------------------------------------------
\subsection{One-point compactification in higher dimensions}

\begin{theorem}[One-point compactification of $\mathbb{R}^n$]\label{thm:onepoint}
For $n\geq 1$, the one-point compactification $\mathbb{R}^n\cup\{\infty\}$ is 
homeomorphic to the $n$-sphere $S^n$, realized by stereographic projection from 
the north pole.
\end{theorem}

\begin{proof}[Proof sketch]
Stereographic projection is a continuous bijection from $\mathbb{R}^n$ onto 
$S^n\setminus\{N\}$, with continuous inverse. Adding $N$ corresponds to the 
point at infinity, and the induced topology matches the one-point 
compactification. See \citet{munkres2000topology} for details.
\end{proof}

\begin{remark}[Bias parameter in higher dimensions]
Just as in Section~\ref{sec:circle}, one can introduce a translation parameter 
$(x,y)\mapsto(x-b,y-c)$ before projection. This has the effect of focusing 
compactification around $(b,c)$ on the plane, giving more “resolution” to features 
near that point while treating other regions as further toward infinity.  
\end{remark}

% ----------------------------------------------------
\subsection{Limits at infinity on the plane}

\begin{theorem}[Limit at infinity on $\mathbb{R}^2$]\label{thm:R2limit}
Let $f:\mathbb{R}^2\to\mathbb{R}^m$ be continuous. Then 
\[
\lim_{\|(x,y)\|\to\infty} f(x,y)=L
\]
exists if and only if the composition
\[
\tilde f := f\circ \mathrm{St}^{-1}: S^2\setminus\{N\}\to \mathbb{R}^m
\]
extends continuously to $N$ with value $L$.
\end{theorem}

\begin{remark}[Directional vs. unified infinity]
In $\mathbb{R}^2$, one may “go to infinity” along infinitely many directions. 
In the compactified model, these correspond to approaching the north pole along 
different meridians. A two-dimensional limit at infinity exists only if all 
approaches converge to the same value. Otherwise, the function has 
direction-dependent behavior at infinity.
\end{remark}

% ----------------------------------------------------
\subsection{Complex case: the Riemann sphere}

\begin{definition}[Riemann sphere]
The extended complex plane 
\[
\hat{\mathbb{C}} = \mathbb{C}\cup\{\infty\}
\]
is identified with the sphere $S^2$ via stereographic projection. The point at 
infinity corresponds to the north pole. 
\end{definition}

\begin{theorem}[Limit at infinity for meromorphic functions]\label{thm:riemann-meromorphic}
Let $f:\hat{\mathbb{C}}\to\hat{\mathbb{C}}$ be meromorphic. Then the following 
are equivalent:
\begin{enumerate}
    \item $\lim_{z\to\infty} f(z)=L$ in $\hat{\mathbb{C}}$.
    \item $f$ extends continuously to $\infty$ on the Riemann sphere with 
    $f(\infty)=L$.
\end{enumerate}
Moreover, $f$ has a pole at $\infty$ if $|f(z)|\to\infty$ as $z\to\infty$.
\end{theorem}

\begin{remark}[Why this matters]
This classification is central in complex analysis:
\begin{itemize}
    \item Polynomials satisfy $f(\infty)=\infty$ (a pole at infinity).
    \item Rational functions either converge to a finite value or diverge at $\infty$.
    \item Entire bounded functions extend continuously to $\infty$ as constants 
    (Liouville’s theorem).
\end{itemize}
Compactification via the Riemann sphere thus provides a unified perspective on 
limits, continuity, and singularities at infinity.
\end{remark}

% ----------------------------------------------------
\subsection{Figures for illustration (to add in v2)}

\begin{itemize}
    \item \textbf{Figure 1:} Circle compactification of $\mathbb{R}$ via tangent 
    half-angle projection, showing how points far out map near the north pole.  
    \item \textbf{Figure 2:} Stereographic projection of $\mathbb{R}^2$ onto $S^2$, 
    with lines from north pole to plane.  
    \item \textbf{Figure 3:} The Riemann sphere depiction, with complex plane drawn 
    at equator and $\infty$ at north pole.  
\end{itemize}

% ====================================================
\section{Implications and Limitations}
\label{sec:discussion}

The compactification framework developed in Sections~\ref{sec:circle} and 
\ref{sec:sphere} has several conceptual and practical implications for analysis, 
geometry, and physics. At the same time, it is important to be clear about the 
scope of the results and where open questions remain.

% ----------------------------------------------------
\subsection{Implications}

\begin{enumerate}
    \item \textbf{Unification of infinities.}  
    On the real line, the traditional distinction between $+\infty$ and $-\infty$ 
    forces separate limit conditions. By embedding $\mathbb{R}$ into $S^1$ via 
    tangent half-angle compactification, these two ends merge into a single 
    point, producing a continuous model of infinity. Similarly, for 
    $\mathbb{R}^2$ and higher dimensions, stereographic projection identifies 
    all directions of escape to infinity with the north pole of the sphere. 
    Limits at infinity are thus reformulated as continuity conditions at a 
    single added point.

    \item \textbf{Classification of behavior at infinity.}  
    The Riemann sphere model provides a clear taxonomy for meromorphic functions 
    at infinity: removable singularities, poles, and essential singularities. 
    This taxonomy is powerful because it translates asymptotic growth into 
    a local property of the compactified space:
    \begin{itemize}
        \item Rational functions extend meromorphically with a finite value or $\infty$ at infinity.  
        \item Polynomials correspond to a pole at infinity.  
        \item Exponential-type functions exhibit essential singularities at infinity.  
    \end{itemize}

    \item \textbf{Geometric interpretation of limits.}  
    Compactification replaces the vague notion of “going further and further 
    away” with the precise operation of approaching a fixed point on a compact 
    space. In higher dimensions, the requirement that all meridians into the 
    north pole converge to the same value explains why many two-dimensional 
    limits fail to exist.

    \item \textbf{Bias and focal control.}  
    Introducing a bias parameter before projection makes it possible to 
    “zoom in” on a region of interest. For instance, shifting the circle or 
    plane before compactification emphasizes resolution near that region while 
    compressing detail elsewhere. This offers flexibility for applications that 
    need directional focus rather than a purely symmetric treatment of infinity.

    \item \textbf{Connections to analysis and physics.}  
    Compactification techniques appear naturally in Fourier analysis (periodic 
    boundary conditions), dynamical systems (circle rotations, ergodic theory), 
    and physics (conformal compactifications of spacetime). The tangent 
    half-angle and stereographic models here provide simple prototypes of these 
    more general constructions.
\end{enumerate}

% ----------------------------------------------------
\subsection{Limitations}

\begin{enumerate}
    \item \textbf{Compactification is not unique.}  
    The one-point compactification is canonical in topology, but other models 
    exist (e.g.\ the extended real line $\overline{\mathbb{R}}$ that preserves 
    separate $+\infty$ and $-\infty$). The appropriate choice depends on 
    context: probability theory often requires directional infinities, while 
    complex analysis prefers a unified infinity.

    \item \textbf{Loss of directional information.}  
    Stereographic projection collapses all escape directions to a single point. 
    This is powerful for continuity arguments, but sacrifices fine-grained 
    directional data about how a function behaves toward infinity.

    \item \textbf{Dependence on scaling and bias.}  
    Using irrational angular scalings (e.g.\ multiples of $\pi$) avoids 
    degenerate periodicities and yields dense coverage of $S^1$, but rational 
    choices can lead to finite orbits. Similarly, introducing a bias parameter 
    changes the apparent “resolution” of infinity, which is useful but also 
    context-dependent. Compactification is therefore not a universal solution, 
    but a tunable tool.

    \item \textbf{Analytic limitations.}  
    The framework is topological in nature. It clarifies the structure of limits 
    at infinity but does not by itself resolve analytic questions of growth, 
    oscillation, or essential singularities. For example, functions with 
    essential singularities at infinity remain unpredictable even when classified 
    neatly on the Riemann sphere.
\end{enumerate}

% ----------------------------------------------------
\subsection{Summary}
Compactifying infinity via circles, spheres, and the Riemann sphere provides a 
unified and geometrically intuitive framework. It clarifies when limits exist, 
offers a natural classification for complex functions, and connects to 
applications across mathematics and physics. At the same time, it collapses 
directional information and depends on context-specific choices of scaling or 
bias. These trade-offs mark both the power and the boundaries of the approach.

% ====================================================
\section{Conclusion}

This paper has shown how the real line, the plane, and the complex plane can be 
compactified by mapping them onto the circle, the sphere, and the Riemann sphere, 
respectively. Through tangent half-angle and stereographic projection, we proved 
that limits at infinity exist precisely when functions extend continuously to the 
added point at infinity. This provides a unified framework in which $+\infty$ and 
$-\infty$ on $\mathbb{R}$, or infinitely many escape directions on $\mathbb{R}^2$, 
are collapsed into a single continuous notion of infinity.  

The benefit of this approach is twofold: it clarifies when limits and continuity 
at infinity are well-defined, and it translates asymptotic growth into local 
behavior on a compact manifold. The framework further yields a geometric 
classification of functions in complex analysis, highlighting how removable 
singularities, poles, and essential singularities at infinity naturally arise on 
the Riemann sphere.  

Future work will explore how compactification interacts with frequency and 
equidistribution results (e.g.\ irrational rotations on $S^1$), how PDEs behave 
on compactified domains, and how conformal compactification methods can inform 
both dynamical systems and physical models such as quantum mechanics and 
spacetime geometry.

% ====================================================
% REFERENCES (add in final; placeholder comment here)
% - Topology: Munkres, Hatcher (one-point compactification).
% - Complex Analysis: Ahlfors, Stein–Shakarchi, Needham (Riemann sphere, stereographic).
% - Equidistribution: Weyl 1916; Kuipers–Niederreiter (uniform distribution sequences).
% - Dynamical systems: Katok–Hasselblatt (irrational rotations).
% Use your shared .bib across the series for consistency.

\end{document}