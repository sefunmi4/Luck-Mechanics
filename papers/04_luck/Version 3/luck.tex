\documentclass[12pt]{article}
\usepackage[margin=1in]{geometry}
\usepackage{amsmath, amssymb, amsthm, bm}
\usepackage{mathtools}
\usepackage{physics}
\usepackage{siunitx}
\usepackage{graphicx}
\usepackage{caption}
\usepackage{subcaption}
\usepackage[numbers]{natbib}
\usepackage{xcolor}
\usepackage{booktabs}
\usepackage{enumitem}
\usepackage{filecontents}
\usepackage{hyperref}
\hypersetup{colorlinks=true,linkcolor=blue,citecolor=blue,urlcolor=blue}

\title{Luck Mechanics: A Unified Framework for Probabilistic Wavefunctions,\\
Compactification, Prime Interference, Attention Dynamics, and Polar Spacetime Optics in the Omniverse}
\author{Sefunmi Ashiru}
\date{\today}


\newcommand{\Luck}{\mathcal{L}}
\newcommand{\Opp}{\mathcal{O}}
\newcommand{\Prep}{\mathcal{P}}
\newcommand{\Circ}{\mathcal{C}}
\newcommand{\Afield}{\mathcal{A}}
\newcommand{\phiang}{\phi}
\newcommand{\Psit}{\tilde{\Psi}}
\newcommand{\Sphere}{S^2}
\newcommand{\Circle}{S^1}

\newcommand{\Primes}{\mathbb{P}}
\newcommand{\N}{\mathbb{N}}
\newcommand{\Z}{\mathbb{Z}}

\newcommand{\Stereo}{\Phi}
\newcommand{\LuckFormula}{\Luck = (\Opp \cdot \Prep) + \Circ}

\theoremstyle{plain}
\newtheorem{theorem}{Theorem}
\newtheorem{proposition}{Proposition}
\newtheorem{lemma}{Lemma}
\theoremstyle{definition}
\newtheorem{definition}{Definition}

\begin{document}


\maketitle

\begin{abstract}
  We introduce \emph{Luck Mechanics}, a framework that treats ``luck'' as the tensioned energy of frequency permeating the universe—from particle oscillations to gravitational waves. In this view, energy is frequency across scales, while gravity appears as the manifestation of this tension force. Black and white holes act as spectral gates, retuning harmonics and shaping the flow of information.  
  
  Matter is reinterpreted as compactified light: outward-radiating frequencies stabilize into particles through tension, forming persistent ``luck force.'' This same tension underlies gravity across atomic to cosmic domains. Einstein’s relation \(E=mc^2\) extends to the full relativistic form
  \[
  E^2 = (mc^2)^2 + (pc)^2,
  \]
  where momentum (\(p\)) accounts for dynamical contributions. In parallel, Luck Mechanics formalizes its canonical law as
  \[
  \Luck = (\Opp \cdot \Prep) + \Circ,
  \]
  where opportunity (\(\Opp\)) is probabilistic, preparation (\(\Prep\)) is deterministic, and context (\(\Circ\)) represents the angular measurement of events within a light/luck cone on a four-dimensional hyperspectral plane radiating from an origin point.  
  
  At the Planck scale, polar reparameterization—time as radial, length as angular—visualizes spacetime as a continuous expansion where angular sectors represent different perspectives of a single universe. The so-called multiverse is thus reframed as directional viewpoints within one connected cosmos.  
  
  Luck Mechanics synthesizes compactification, interference, frequency-selective amplification, and polar optics into a unified framework, positioned alongside string vibration, entropic gravity, and frequency–tension models. We conclude with testable predictions and prototypes that treat luck as a guiding principle for directing probabilistic outcomes across physics and cognition.
\end{abstract}

\tableofcontents
\newpage

\section{Introduction}

\emph{Luck Mechanics} proposes that many seemingly disparate phenomena—probability, attention, interference, inertia, and gravitation—emerge from a single substrate: a frequency–tension field whose amplitudes and phase gradients shape dynamics across scales. In this view, energy is frequency, tension is curvature-inducing phase gradient, and “luck’’ is the directed amplification of compatible frequencies under constraints of preparation and context. The framework is distilled in the operational law
\[
\Luck = (\Opp \cdot \Prep) + \Circ,
\]
where \emph{opportunity} $\Opp$ encodes probabilistic affordances, \emph{preparation} $\Prep$ encodes learned or engineered priors, and \emph{context} $\Circ$ captures situation-specific boundary conditions. We treat $\Opp,\Prep,\Circ$ as gain-like fields acting on spectral representations, so that outcomes correspond to amplitude selection in an underlying oscillatory medium.

\paragraph{Positioning.}
Our approach sits alongside three broad families of ideas: (i) \emph{frequency–tension models} that take an oscillatory medium as fundamental \citep{STheory}; (ii) \emph{string/vibrational} pictures in which particles are modes and energy is frequency \citep{Backreaction}; and (iii) \emph{entropic/information} accounts of gravity that tie macroscopic forces to microscopic state counting and information flow \citep{EntropicGravity}. We adopt a pragmatic synthesis: treat frequency as the carrier, tension (phase gradient) as the driver of effective curvature, and probabilistic selection as amplitude modulation governed by $\Opp,\Prep,\Circ$.

\paragraph{Geometry at the Planck scale.}
Motivated by natural units $t_P$ and $\ell_P$ \citep{PlanckUnits}, we develop a polar reparameterization in which time is radial ($r=ct$) and spatial directions are angular sectors. This chart makes causal flow explicit and converts multiverse-like branching into \emph{angular} viewpoints on one connected history, simplifying continuity conditions at small scales \citep{PolarCoords}. We use this polar optic to reinterpret horizons (black/white holes) as \emph{spectral gates} that retune phases and redistribute amplitudes across sectors, with phenomenology constrained by relativistic uncertainty at high curvature \citep{PhysRevResearch033343}.

\paragraph{From probability to mechanics.}
In Luck Mechanics, probabilities arise as \emph{spectral gains}: interference patterns and attention-like operators select bands aligned with opportunity, modulated by preparation and context. This connects laboratory optics (interference), computational attention (amplitude gating), and macroscopic dynamics (effective curvature) within one calculational scheme. The law $\Luck=(\Opp\cdot\Prep)+\Circ$ thus functions as a control equation over gains on a shared frequency geometry.

\paragraph{Prior scaffolding.}
This work consolidates three article-scale threads by the author(s): compactification of infinities via stereographic maps (circle/sphere/Riemann-sphere models), a wavefunction sieve for arithmetic interference, and a “frequencies-first’’ account of attention as multiplicative gain in the Fourier domain. While each remains pre-journal, they provide the mathematical and conceptual scaffolding used here.

\paragraph{Contributions.}
\begin{itemize}[leftmargin=1.1em]
  \item A polar frequency–tension formalism in which phase gradients play the role of stress sources for emergent curvature, with horizons modeled as spectral gates.
  \item A probabilistic control law for outcomes, $\Luck=(\Opp\cdot\Prep)+\Circ$, realized as amplitude selection on spectral fields and tied to interference/attention operators.
  \item A compactification-based visualization of Planck-scale structure (time radial, space angular) clarifying continuity, branching, and limit behavior \citep{PlanckUnits,PolarCoords}.
  \item Testable predictions linking (i) interferometric amplification in engineered luck-cones, (ii) membrane/optical analogs of curvature via phase-tension, and (iii) attention-as-gain signatures in computational systems.
\end{itemize}

\paragraph{Organization.}
Section~2 introduces notation and the polar frequency–tension geometry. Section~3 formalizes the $\Luck$ control law and its operators. Section~4 presents experimental/engineering prototypes and predictions. Section~5 discusses implications and limitations, including relations to string/entropic and frequency–tension programs \citep{STheory,Backreaction,EntropicGravity}. Section~6 concludes.
\subsection{Related Work}
A number of recent proposals have sought to bridge quantum mechanics and general relativity through geometric or frequency-based approaches.  
Cheok, Nardelli, Srichan, and collaborators \citep{Cheok2025} suggest that both quantum physics and Einstein’s equations can be derived from a shared Riemannian and Planck-scale formalism, where mass and charge are emergent from curvature–energy interactions. While their framework is bold, it has faced critical scrutiny.  

Luck Mechanics differs in emphasis: rather than starting from curvature tensors, we take \emph{frequency–tension} as the universal substrate. Compactification stabilizes infinities, prime interference encodes arithmetic cancellation, and attention kernels modulate probability distributions. This offers a mathematically transparent and experimentally approachable bridge, complementary to curvature-based approaches.

\paragraph{Organization.}  
Section~2 introduces notation and preliminaries. Section~3 develops the cross-domain synthesis. Section~4 proposes experimental predictions. Section~5 discusses implications and limitations. Section~6 concludes.


\section{Unified Notation and Preliminaries}

To ground the framework of Luck Mechanics, we introduce standard notation for the 
\emph{luck field}, its polar representation at the Planck scale, and the 
compactification–interference constructions that underpin its arithmetic structure.  

% ----------------------------
\subsection{Luck field}

\begin{definition}[Luck field]
Let $\Opp,\Prep,\Circ : \mathbb{R}^d \times \mathbb{R} \to \mathbb{R}$ denote 
the \emph{opportunity}, \emph{preparation}, and \emph{context} functions, 
respectively.  
The \emph{luck field} is defined by
\[
\Luck(x,t) \;=\; \Opp(x,t)\,\Prep(x,t) + \Circ(x,t).
\]
Here $\Opp$ is interpreted as a probabilistic affordance, $\Prep$ as a 
deterministic amplifier, and $\Circ$ as the contextual offset contributed by 
local measurement or boundary conditions.  
\end{definition}

This functional form encodes the canonical law 
$\Luck = (\Opp \cdot \Prep) + \Circ$ introduced in the abstract, treating outcomes 
as multiplicative–additive combinations of these spectral components.

% ----------------------------
\subsection{Polar representation}

At the Planck scale, we employ a radial–angular reparameterization of spacetime 
to visualize continuity and branching. Time is mapped to a radial coordinate 
$r = ct$, while spatial extension along a bounded interval is mapped to an 
angular coordinate.

\begin{definition}[Polar mapping]
For $x \in [-L/2,L/2]$ and $t \ge 0$, define
\[
r = c t, 
\qquad 
\theta(x) = 2\pi \frac{x + L/2}{L},
\]
and the polar-transformed wavefunction
\[
\Psit(r,\theta) 
\;=\; \sqrt{\tfrac{2\pi}{L}}\, \Psi\!\bigl(x(\theta), t\bigr).
\]
\end{definition}

This mapping projects $\Psi(x,t)$ onto a radial–angular domain, where radial 
expansion represents temporal evolution and angular sectors represent directional 
perspectives of the same universe.

% ----------------------------
\subsection{Compactification and prime wave-sieve}

Compactification replaces unbounded domains with closed manifolds, allowing 
infinity to be treated as a single point. Following \citet{valvo2024limits}, 
we use stereographic projection to identify
\[
\mathbb{R} \;\to\; S^1,
\qquad
\mathbb{R}^2 \;\to\; S^2.
\]

Within this geometry, prime divisibility can be modeled via periodic 
wavefunctions. For each prime $p$ define
\[
\psi_p(n) = \sin\!\left(\tfrac{2\pi n}{p}\right).
\]
The adaptive product
\[
\Phi(n) = \prod_{p \le \sqrt{n}} \psi_p(n)
\]
vanishes precisely when $n$ is composite, since some factor $\psi_p(n)$ 
cancels whenever $p \mid n$.  
As shown in \citet{ashiru2024wave}, this construction yields an 
interference-based arithmetic sieve, linking number-theoretic structure to 
wave superposition.
\section{Cross-Domain Synthesis: Emerging Harmony}

The central aim of Luck Mechanics is to merge compactification, interference, 
polar reparameterization, and attention into one coherent frequency–tension 
framework. In this section we highlight how these pieces interact and reinforce 
each other across domains.

% ----------------------------
\subsection{Compactification meets interference}

Compactification replaces the two disjoint infinities $+\infty$ and $-\infty$ 
with a single continuous point. When coupled with the prime wave-sieve 
construction, this geometry produces dense angular distributions of residues 
on $S^1$, consistent with irrational rotation equidistribution 
(Weyl, Kronecker). In this view, arithmetic cancellation and geometric wrapping 
become two manifestations of the same interference principle.

% ----------------------------
\subsection{Luck cones in polar geometry}

Mapping the wavefunction $\Psi$ into radial–angular coordinates generates 
\emph{luck cones}: angular sectors of overlapping polar trajectories that act 
as probabilistic amplifiers. These cones represent regions where distinct 
trajectories interfere constructively, amplifying certain outcomes. They 
connect compactified infinities to many-worlds style branching, but in a 
geometry where branches are not separate universes, only angularly displaced 
perspectives within one connected field.

% ----------------------------
\subsection{Attention as frequency modulation}

As shown in \citet{ashiru2024frequencies}, attention can be modeled as a 
multiplicative gain in Fourier space. Within Luck Mechanics, this gain becomes 
the operator that foregrounds particular trajectories inside a luck cone, 
effectively selecting which angular paths dominate experience. Attention thus 
emerges as frequency-selective amplification: probabilistic trajectories are 
always present, but focus makes some outcomes more likely than others.

% ----------------------------
\subsection{Polar coordinates and unified fields}

Recasting Planck time as a radial axis and Planck length as an angular domain 
\citep{PlanckUnits,PolarCoords} provides a natural visualization of the 
continuous structure of spacetime. In this representation, field evolution 
unfolds across radial–angular sectors, explicitly linking temporal flow with 
spatial extension.  

This polar mapping unifies quantum field descriptions with gravitational 
curvature, while also offering a lens for modeling horizons: black holes and 
white holes appear as angular retunings of the same radial field. Within Luck 
Mechanics, all physical media are reinterpreted as excitations of a single 
frequency–tension substrate—a \emph{luck field}—whose oscillations encode 
probabilistic outcomes. Overlapping light-cone sectors create interference 
regions, or ``luck cones,'' echoing many-worlds interpretations while remaining 
grounded in a unified frequency geometry. Comparable constructions appear in 
phase-space treatments of cosmological ensembles \citep{GibbonsTurok2006}. 
Constraints at the Planck scale further suggest frequency/angle–tension couplings 
consistent with relativistic uncertainty bounds \citep{PhysRevResearch033343}.

% ----------------------------
\subsection{Polar optics and emergent gravity}

Finally, consider a complex field $\Afield = |\Afield|e^{i\phiang}$. Phase 
gradients induce tension analogous to stress–energy, generating curvature-like 
effects. This links frequency–tension models directly to emergent gravity: 
spacetime curvature is reinterpreted as the macroscopic manifestation of 
underlying phase–tension dynamics in the luck field.

\section{Experimental Predictions}

The Luck Mechanics framework suggests several concrete experimental avenues, 
ranging from tabletop prototypes to conceptual analogs in optics and mechanics. 
Each test probes whether frequency–tension principles can reproduce or illuminate 
features usually attributed to spacetime curvature, probability, or attention.

\begin{itemize}[leftmargin=1.1em]
  \item \textbf{Mechanical/optical sieve.}  
  Construct oscillatory masks—mechanical membranes or optical gratings—whose 
  periodic nodes replicate the interference structure of the prime wave-sieve. 
  By illuminating or vibrating the mask, one should observe cancellation bands 
  at composite positions and residual amplification at “prime-like” sites. This 
  provides a direct visualization of arithmetic encoded as wave interference.

  \item \textbf{Angular interferometry.}  
  Develop interferometric setups where overlapping beams or waves are passed 
  through angular phase shifters. The goal is to detect selective amplification 
  in regions corresponding to ``luck cones’’—interference sectors where phase 
  alignment amplifies certain trajectories. This experiment tests the principle 
  that probabilistic outcomes can be steered by angular retuning of frequencies.

  \item \textbf{Macro tension fields.}  
  Engineer resonant membranes or elastic lattices whose phase gradients induce 
  measurable curvature-like deformations. These analog systems act as 
  large-scale models of the frequency–tension substrate. If localized stress or 
  oscillation produces curvature-like focusing of trajectories, it would provide 
  a tangible prototype for emergent gravity within the Luck Mechanics framework.
\end{itemize}

\section{Discussion \& Limitations}

\paragraph{Synthesis achieved.}  
Luck Mechanics weaves together several previously separate ideas into a single 
frequency–tension framework:
\begin{enumerate}[leftmargin=1.1em]
  \item Compactification of infinities into smooth geometric continuity, removing 
  the distinction between $+\infty$ and $-\infty$,
  \item Prime interference as exact arithmetic cancellation, linking number theory 
  with wave mechanics,
  \item Attention as spectral amplification, reframed as a frequency-selective gain 
  operator,
  \item Polar coordinates as a unifying map of Planck-scale domains into radial–angular 
  geometry, and
  \item Polar optics as emergent curvature, where phase gradients induce 
  gravity-like tension fields.
\end{enumerate}
Together, these elements suggest that probability, geometry, and curvature may be 
different manifestations of one underlying frequency–tension substrate.

\paragraph{Relation to prior approaches.}  
Unlike curvature-first frameworks such as Cheok \emph{et al.}~\citep{Cheok2025}, 
which begin with Riemannian tensors and field equations, Luck Mechanics emphasizes 
a polar compactification at the Planck scale \citep{PlanckUnits}. By treating 
Planck time as radial and Planck length as angular \citep{PolarCoords}, spacetime 
is visualized as a continuous radial expansion with angular ``retunings'' that 
replace singularities. This perspective offers a more geometric and interference-based 
lens on phenomena usually modeled by curvature alone, and it connects naturally to 
emergent-gravity analogies in condensed matter and optics \citep{PhysRevResearch033343}.

\paragraph{Limitations.}  
The present framework remains speculative and preliminary. Several important 
limitations must be acknowledged:
\begin{itemize}[leftmargin=1.1em]
  \item \textbf{Standard Model gap:} The formulation does not yet recover the 
  particle content or gauge symmetries of the Standard Model.
  \item \textbf{Directional loss:} Compactification collapses directional 
  information at infinity, which may obscure anisotropic features relevant to 
  cosmology.
  \item \textbf{$n$-dependence:} The prime wave-sieve is inherently $n$-dependent, 
  preventing it from being a global analytic function over all integers.
  \item \textbf{Cognitive oversimplification:} The model of attention as pure 
  multiplicative gain neglects recurrent and nonlinear dynamics known to be 
  essential in real neural systems.
\end{itemize}

\paragraph{Future work.}  
Several natural extensions can address these limitations:
\begin{itemize}[leftmargin=1.1em]
  \item \textbf{Spatiotemporal frequency:} Incorporating motion-energy operators 
  may capture relativistic effects and dynamical fields more faithfully.
  \item \textbf{Number-theoretic links:} Connections to Dirichlet characters, 
  Ramanujan sums, and L-function zeros may sharpen the arithmetic side of the 
  framework.
  \item \textbf{Experimental analogues:} Neuromodulation studies and optical 
  interferometry could provide prototypes for the spectral gain operators and 
  luck-cone interference predicted here.
\end{itemize}
Ultimately, the value of Luck Mechanics lies not in replacing established 
theories, but in offering a cross-domain synthesis that opens new conceptual and 
experimental pathways.

\section{Conclusion}

Luck Mechanics reframes probability, interference, and focus as manifestations 
of a single frequency–tension field. Within this framework, compactification 
merges infinities into continuous geometry, prime interference encodes exact 
arithmetic cancellation, and frequency-based attention acts as a spectral gain 
mechanism. Together these elements unify into a polar–optical picture of the 
universe: a jelly-like frequency network in which ``luck'' appears as measurable 
interference across scales, from the quantum to the cosmological.

By positioning luck as a structured outcome of frequency tension rather than a 
mere abstraction, this work suggests new ways to bridge physics, number theory, 
and cognition. While speculative, the framework offers a conceptual synthesis 
that motivates experimental analogues, computational prototypes, and deeper 
mathematical connections. In this sense, Luck Mechanics is not an alternative to 
established theories but a complementary lens—one that highlights interference, 
compactification, and probabilistic amplification as unifying principles across 
domains.

\section*{Acknowledgments}
I am profoundly grateful to my parents and family—and especially to my mother—
for their unwavering love, encouragement, and support throughout this journey.  
I would like to thank Dr.~Scales for introducing me to the field through early 
lab work on quantum photonics, which provided a vital foundation for much of my 
subsequent research.  
I am also indebted to Daniel Valvo for formative discussions on compactification 
and sieves, and to my colleagues for their valuable insights into attention, 
frequency-based models, and the broader development of this framework.

\bibliographystyle{plainnat}
\bibliography{references}

\end{document}