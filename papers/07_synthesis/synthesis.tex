% ============================================================
% LUCK MECHANICS — FLAGSHIP SYNTHESIS PAPER (PAPER 7)
% One document to unify Papers 1–6; built for arXiv/journal portability.
% Keep comments; they enforce consistency across the series.
% ============================================================

\documentclass[12pt]{article}

% -------------------------------
% Packages (portable + journal-safe)
% -------------------------------
\usepackage[margin=1in]{geometry}
\usepackage{amsmath, amssymb, amsthm, bm}
\usepackage{mathtools}
\usepackage{physics}
\usepackage{siunitx}
\usepackage{graphicx}
\usepackage{caption}
\usepackage{subcaption}  % For multi-panel figs A/B/C (reusing figs from Papers 1–6)
\usepackage{hyperref}
\usepackage[numbers]{natbib}
\usepackage{xcolor}
\usepackage{booktabs}
\usepackage{enumitem}
\hypersetup{colorlinks=true,linkcolor=blue,citecolor=blue,urlcolor=blue}

% -------------------------------
% Metadata
% -------------------------------
\title{Luck Mechanics: A Unified Framework for Probabilistic Wavefunctions,\\
Compactification, Prime Interference, Attention Dynamics, and Polar Spacetime Optics}
\author{Sefunmi Ashiru}
\date{\today}

% -------------------------------
% Global Macros (LOCK THESE; reuse across all papers)
% -------------------------------
% Core fields & objects
\newcommand{\Luck}{\mathcal{L}}      % Luck field
\newcommand{\Opp}{\mathcal{O}}       % Opportunity field
\newcommand{\Prep}{\mathcal{P}}      % Preparation/Action field
\newcommand{\Circ}{\mathcal{C}}      % Prepared Circumstance (bias)
\newcommand{\Afield}{\mathcal{A}}    % Complex field A = |A| e^{i\phi}
\newcommand{\phiang}{\phi}           % Phase
\newcommand{\Psit}{\tilde{\Psi}}     % Polar-mapped wavefunction
\newcommand{\Sphere}{S^2}
\newcommand{\Circle}{S^1}

% Number theory
\newcommand{\Primes}{\mathbb{P}}
\newcommand{\N}{\mathbb{N}}
\newcommand{\Z}{\mathbb{Z}}

% Compactification map (stereographic projection)
\newcommand{\Stereo}{\Phi}

% Paper cross-references (DO NOT RENUMBER; use same names everywhere)
\newcommand{\PaperI}{Paper~1 (Limits to Complex Infinities)}
\newcommand{\PaperII}{Paper~2 (Wavefunction Sieve of Eratosthenes)}
\newcommand{\PaperIII}{Paper~3 (Prime Superposition \& Coverage)}
\newcommand{\PaperIV}{Paper~4 (Attention \& Frequency Models)}
\newcommand{\PaperV}{Paper~5 (Polar Ripple Framework)}
\newcommand{\PaperVI}{Paper~6 (Spacetime as Polar Optics)}

% Luck formula (canonical)
\newcommand{\LuckFormula}{\Luck = (\Opp \cdot \Prep) + \Circ}

% Theorem environments (consistent phrasing)
\theoremstyle{plain}
\newtheorem{theorem}{Theorem}
\newtheorem{proposition}{Proposition}
\newtheorem{lemma}{Lemma}
\theoremstyle{definition}
\newtheorem{definition}{Definition}
\theoremstyle{remark}
\newtheorem{remark}{Remark}

% Figure panel labels
\newcommand{\figlabel}[1]{\textbf{(#1)}}

% -------------------------------
\begin{document}
\maketitle

% ============================================================
\begin{abstract}
% 150–250 words. Keep it crisp:
% 1) Problem/context; 2) Method (unified framework); 3) Key results from Papers 1–6; 4) Implications.
% Avoid speculative claims; tie to quantifiable statements and cross-references.
This flagship paper synthesizes \PaperI–\PaperVI\ into a single framework, \emph{Luck Mechanics}, which treats chance as a quantifiable interference field across mathematics, physics, and cognition. We formalize the canonical relationship \(\LuckFormula\) and establish: (i) compactification of infinite possibility spaces onto curved finite domains (circles, spheres, Riemann sphere); (ii) prime periodicity as sinusoidal interference and harmonic coverage; (iii) frequency-selective attention as probabilistic amplification; (iv) a polar radial–angular representation for quantum dynamics with luck cones as overlapping light cones; and (v) a polar-optics ansatz linking phase gradients to curvature. We present a unified notation, derive inter-paper equivalences, and outline falsifiable predictions and algorithmic prototypes. This synthesis sets a consistent foundation for continued study and applied tests.
\end{abstract}

\tableofcontents
\newpage

% ============================================================
\section{Introduction}
% PURPOSE:
% - State the unified goal and why unification matters.
% - Give 1–2 sentences per prior paper, referencing their unique contribution.
% - Finish with a bullet roadmap of flagship contributions (unification lemmas, shared notation, cross-derivations, tests).
Luck Mechanics proposes that probabilistic structure is geometrically and harmonically representable across domains. Building on \PaperI–\PaperVI, we consolidate key definitions, theorems, and constructions into one coherent scheme. The contributions of this synthesis are:
\begin{itemize}[leftmargin=1.1em]
  \item A single, field-theoretic statement of \(\LuckFormula\) with sinusoidal carriers and bias.
  \item A compactification engine for infinite limits (\PaperI) interfacing with wave-sieve arithmetic (\PaperII, \PaperIII).
  \item A shared polar radial–angular geometry (\PaperV) that hosts luck cones and connects to spacetime optics (\PaperVI).
  \item A cross-domain attention kernel formalism (\PaperIV) for brains/AI as frequency-selective probabilistic amplifiers.
  \item Falsifiable predictions, algorithmic prototypes, and reproducibility standards unifying the series.
\end{itemize}

% ============================================================
\section{Unified Notation and Preliminaries}
% PURPOSE:
% - Freeze notation used across the series; eliminates renaming pain during compilation of results.
% - Put core definitions here so later sections can cite them.
\subsection{Core fields and domains}
\begin{definition}[Luck field]
Let \(\Opp,\Prep,\Circ:\mathbb{R}^{d}\times\mathbb{R}\to\mathbb{R}\) be measurable fields. The \emph{luck field} is
\[
\Luck(x,t) \coloneqq \Opp(x,t)\,\Prep(x,t) + \Circ(x,t).
\]
% Comment: If needed, allow vector-valued \Opp, \Prep with \(\cdot\) as inner product.
\end{definition}

\begin{definition}[Polar mapping for dynamics]
For a spatial coordinate \(x\in[-L/2,L/2]\) and time \(t\ge0\), define \(r=ct\), \(\theta(x)=2\pi(x+L/2)/L\), and the polar-mapped state
\[
\Psit(r,\theta) \coloneqq \sqrt{\frac{2\pi}{L}}\,\Psi(x(\theta),t).
\]
% Comment: This is the canonical map from \PaperV; keep exactly this normalization.
\end{definition}

\subsection{Compactification maps}
We adopt stereographic projection \(\Stereo:\mathbb{C}\cup\{\infty\}\to \Sphere\) as in \PaperI, ensuring finite-point analysis at infinity.

\subsection{Prime wavefunctions}
For prime \(p\in\Primes\), define \(\psi_p(n)=\sin(2\pi n/p)\) and composite-eliminating product \(\Psi_P(n)=\prod_{p\le P}\psi_p(n)\) (\PaperII).

% ============================================================
\section{Synthesis I: Compactification \texorpdfstring{(\PaperI)}{} and Harmonic Arithmetic \texorpdfstring{(\PaperII,\PaperIII)}{}}
% PURPOSE:
% - Show how compactification stabilizes limits for the wave-sieve.
% - Prove “glue lemmas” that directly connect Paper 1 to Papers 2–3.

\subsection{Compactified limits for probabilistic ensembles}
\begin{proposition}[Finite-point limit at infinity]
Under \(\Stereo\), limits \(z\to\infty\) correspond to approach to the north pole on \(\Sphere\), preserving angle structure and enabling well-posed probabilistic limits.
\end{proposition}
\begin{proof}
% Keep short; cite \PaperI for full proof details.
Sketch following \PaperI: stereographic projection is conformal; limits at \(\infty\) map to a finite point while preserving local angles, ensuring measure-theoretic stability.
\end{proof}

\subsection{Wave-sieve as interference on \texorpdfstring{\Circle}{}}
\begin{theorem}[Composite cancellation via prime product]
For \(n\in\N\), \(\Psi_P(n)=0\) if and only if \(n\) shares a factor with some \(p\le P\). Surviving amplitudes correspond to primes \(>P\) or unresolved candidates.
\end{theorem}
\begin{proof}
Immediate from zeros of \(\sin(2\pi n/p)\) at \(n=kp\) (\PaperII).
\end{proof}

\subsection{Coverage ratios and thinning}
\begin{proposition}[Effective coverage contribution]
The net new elimination fraction from prime \(p\) is \(f(p)=\frac{1}{p}\prod_{q<p}(1-1/q)\).
\end{proposition}
\begin{proof}
Inclusion–exclusion argument as formalized in \PaperIII.
\end{proof}

% ============================================================
\section{Synthesis II: Attention and Frequency Kernels \texorpdfstring{(\PaperIV)}{}}
% PURPOSE:
% - Present a minimal operator formalism that other sections can call.
\begin{definition}[Attention operator]
Given a signal \(f\) with transform \(\hat f(\omega)\), define
\[
\mathcal{A}_K[f](t)=\int K(\omega)\,\hat f(\omega)\,e^{i\omega t}\,d\omega,
\]
where \(K(\omega)\) is a (possibly adaptive) gain kernel.
\end{definition}
\begin{proposition}[Selective amplification of probabilistic channels]
If \(K\) concentrates around carrier \(\omega^\star\), then \(|\mathcal{A}_K[f](t)|^2\) amplifies events phase-aligned with \(\omega^\star\), yielding an increase in measured luck \(|\langle \Psit,\mathcal{A}_K\Psit\rangle|\).
\end{proposition}
\begin{proof}
Fourier-domain gain control; see \PaperIV for constructions.
\end{proof}

% ============================================================
\section{Synthesis III: Polar Ripple Quantum Geometry \texorpdfstring{(\PaperV)}{} and Luck Cones}
% PURPOSE:
% - Define luck cones (overlapping light cones) rigorously in polar coordinates.
\begin{definition}[Luck cone]
On a radial slice \(r\), define a sector \(\Theta\subset[0,2\pi)\) as a \emph{luck cone} if
\[
\int_{\Theta} |\Psit(r,\theta)|^2\,d\theta \ge \tau(r),
\]
for threshold \(\tau(r)\) determined by instrument/medium. Overlaps of light-cone-sourced fields elevate \(|\Psit|^2\) within \(\Theta\).
\end{definition}

% ============================================================
\section{Synthesis IV: Spacetime as Polar Optics \texorpdfstring{(\PaperVI)}{}}
% PURPOSE:
% - Present the unification ansatz, but keep it compatible with GR/QFT limits.
\subsection{Action and stress}
We consider
\[
S[\Afield,g]=\int d^4x\,\sqrt{-g}\Big(\tfrac{1}{2\kappa}R+\tfrac{\xi}{2}\nabla_\mu \Afield \nabla^\mu \Afield^* - V(|\Afield|)\Big),
\]
with \(\Afield=|\Afield|e^{i\phiang}\) and phase-stress \(T^{(\phi)}_{\mu\nu}=\xi(\nabla_\mu \phiang \nabla_\nu \phiang - \tfrac12 g_{\mu\nu}(\nabla \phiang)^2)\).

\subsection{Consistency limits}
% SHORT bullets (reviewers love this):
\begin{itemize}[leftmargin=1.1em]
  \item \textbf{GR limit:} small phase gradients recover standard scalar sources in Einstein’s equations.
  \item \textbf{QFT limit:} on fixed \(g\), \(\Afield\) obeys curved-space Klein–Gordon dynamics.
  \item \textbf{SM compatibility:} treat \(\Afield\) as an effective sector; precision constraints bound couplings.
\end{itemize}

% ============================================================
\section{Equivalences and Cross-Domain Lemmas}
% PURPOSE:
% - The "glue" section for the flagship. Offer 2–4 precise statements relating domains.
\begin{lemma}[Compactified arithmetic \(\leftrightarrow\) attention kernels]
Mapping integer registers to \(\Circle\) (via \PaperI) and applying band-limited \(K(\omega)\) selects residue classes analogous to modular filters in \PaperII–\PaperIII.
\end{lemma}

\begin{lemma}[Luck cones \(\leftrightarrow\) harmonic survival]
Sectors with elevated \(|\Psit|^2\) correspond to survival sets of the interference sieve under appropriate identification of carriers; see \PaperV with \PaperII.
\end{lemma}

% ============================================================
\section{Predictions, Algorithms, and Experiments}
% PURPOSE:
% - Concrete tests + code pointers. Keep it terse and actionable.
\subsection{Optical/superfluid prime-mask test}
% Reuse description from Papers 2–3; here, specify metrics and expected nulls.
\subsection{Angular interferometry (polar tomography)}
% Tie to \PaperV; specify reweighting across \theta and GR-consistent limits.
\subsection{Compact-circle arithmetic prototype}
% Provide pseudocode, complexity notes, and evaluation plan.

% ============================================================
\section{Limitations and Validity Domain}
% PURPOSE:
% - Proactively address reviewer concerns. Distinguish proven parts vs. conjectures.
\begin{itemize}[leftmargin=1.1em]
  \item Proven: composite cancellation, coverage ratios, compactified limits, attention-operator gains.
  \item Model-dependent: mapping attention kernels to neurophysiology; gravity-as-optics couplings.
  \item Out-of-scope: derivation of SM spectrum; claims conflicting with Bell constraints without explicit nonlocality/contextuality.
\end{itemize}

% ============================================================
\section{Conclusion}
% PURPOSE:
% - One paragraph: what is unified, what is measured, what is next.
We unified compactified probability, prime interference, attention kernels, polar quantum geometry, and a polar-optics ansatz under a single notation and set of glue lemmas. The measurable heart of Luck Mechanics is interference-driven survival with tunable bias, expressed as \(\LuckFormula\). The next step is empirical: run mask/tomography prototypes and benchmark compact-circle arithmetic, then iterate couplings subject to GR/QFT precision.

% ============================================================
\section*{Reproducibility, Data \& Code Availability}
% Link (when ready): GitHub/Zenodo; list scripts for figs; specify versions and seeds.

\section*{Acknowledgments}
% Funding, mentors, readers; acknowledge prior arXiv IDs for Papers 1–6 when posted.

% ============================================================
\bibliographystyle{plainnat}
\bibliography{references} % Use the SAME .bib as Papers 1–6.

% ============================================================
\appendix
\section{Notation Table (Quick Reference)}
% Make a table of all symbols used (Luck, O, P, C, A, Psi~, Phi, etc.), one-liners.

\section{Deferred Proofs and Technical Details}
% Put longer proofs here; in body, keep the synthesis readable.

\section{Figure Panels and Reuse Map}
% Map Fig. numbers in this flagship to originating figs in Papers 1–6.

\end{document}